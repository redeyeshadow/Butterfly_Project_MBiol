\documentclass[a4paper,11pt]{article}
\usepackage{graphicx}
\usepackage[utf8]{inputenc} 
\usepackage{amsmath,amssymb,amsthm} 
\usepackage{booktabs}
\usepackage{hyperref}
\usepackage{longtable}
\usepackage{array}
\usepackage{float}
\usepackage{pdflscape}
\usepackage{microtype}   % nicer justification
\usepackage{caption}

\title{SI for \emph{Museum specimens reveal butterfly size increases through time but divergent responses to temperature}}
\author{Elsa Heywood,  Zoe M. Simmons, J. Christopher D. Terry, Ailsa H. C. McLean}
\date{}

\begin{document}

\maketitle

\renewcommand{\thetable}{S1} % makes this table Table S1
\begin{landscape}
\section{SI Tables}

\footnotesize % shrink table text; change to \small or \scriptsize if needed
\setlength{\tabcolsep}{4pt} % reduce horizontal padding between columns
\renewcommand{\arraystretch}{1.05} % slightly tighter row spacing

\begin{longtable}{@{}
  >{\itshape}p{0.15\linewidth}  % Species (wraps)
  p{0.18\linewidth}  % Common Name (wraps)
  p{0.10\linewidth}  % Family
  p{0.05\linewidth}  % Habitat
  p{0.06\linewidth}  % Foodplant
  r                   % NHM_n (numeric, right-aligned)
  r                   % OUM_n (numeric, right-aligned)
  p{0.15\linewidth}  % Focal Brood (wraps)
  p{0.10\linewidth}  % FocalMonthLateLarval
  p{0.10\linewidth}  % GB Red List category
 @{}}

\caption{Details of butterfly species used in this study. Habitat and foodplant specialism status was split into two categories divided  specialists (S) and generalists (G). Numbers for each datasets are the post-filterring counts. See code repository for corresponding .csv file.}\\
\toprule
Species & Common Name & Family & Habitat & Foodplant & n (NHM) & n (OUMNH) & Focal Brood & Focal Month (Late Larval) & GB Red List category \\
\midrule
\endfirsthead

\multicolumn{10}{c}%
{\tablename\ \thetable\ -- \textit{Continued from previous page}}\\
\toprule
Species & Common Name & Family & Habitat & Foodplant & n (NHM) & n (OUMNH) & Focal Brood & Focal Month (Late Larval) & GB Red List category \\
\midrule
\endhead

\midrule
\multicolumn{10}{r}{\textit{Continued on next page}}\\
\endfoot

\bottomrule
\endlastfoot

Aglais urticae & Small tortoiseshell & Nymphalidae & G & S & 971 & 34 &1: Jun-Aug & May & Not Listed \\
Aricia agestis & Brown argus & Lycaenidae & S & S & 975 & 42 & 2:Aug-Oct & July & Not Listed \\
Celastrina argiolus & Holly blue & Lycaenidae & G & G & 303 & 25 & 2:Aug-Oct & June & Not Listed \\
Lasiommata megera & Wall & Nymphalidae & G & G & 619 & 62 & 2: Aug-Oct &June & Endangered \\
Leptidea sinapis & Wood White & Pieridae & S & S & 791 & 70 & 1: Mar-Jun& JulyBefore & Endangered \\
Lycaena phlaeas & Small copper & Lycaenidae & G & S & 2877 & 65 & 2:Jul-Oct & June & Not Listed \\
Pararge aegeria & Speckled wood & Nymphalidae & G & G & 471 & 38 & 2:Jul-Aug & June & Not Listed \\
Pieris brassicae & Large white & Pieridae & G & S & 344 & 44 & 2:Jul-Sep & June & Not Listed \\
Pieris napi & Green-veined white & Pieridae & G & G & 766 & NA & 2:Jul-Sep & June & Not Listed \\
Pieris rapae & Cabbage white & Pieridae & G & G & 445 & NA & 2: Jul-Sep& June & Not Listed \\
Polygonia c-album & Comma & Nymphalidae & G & G & 466 & 39 & 1: Jun-Aug& June & Not Listed \\
Polyommatus bellargus & Adonis Blue & Lycaenidae & S & S & 4830 & 63 &2: Aug-Sep & July & Vulnerable \\
Polyommatus icarus & Common blue & Lycaenidae & G & G & 3752 & 63 & 2:Aug-Sep & July & Not Listed \\
Aglais io & Peacock & Nymphalidae & G & S & 505 & 80 & Univoltine & June& Not Listed \\
Anthocharis cardamines & Orange tip & Pieridae & G & G & 1661 & 64 &Univoltine & JulyBefore & Not Listed \\
Apatura iris & Purple emperor & Nymphalidae & S & S & 124 & NA &Univoltine & March & Not Listed \\
Aphantopus hyperantus & Ringlet & Nymphalidae & G & G & 2206 & NA &Univoltine & June & Not Listed \\
Argynnis paphia & Silver washed fritillary & Nymphalidae & G & S & 1447& 83 & Univoltine & June & Not Listed \\
Boloria euphrosyne & Pearl-bordered Fritillary & Nymphalidae & G & S &1865 & NA & Univoltine & April & Vulnerable \\
Boloria selene & Small Pearl-bordered Fritillary & Nymphalidae & S & S &1925 & NA & Univoltine & April & Vulnerable \\
Callophrys rubi & Green hairstreak & Lycaenidae & G & G & 990 & NA &Univoltine & JuneBefore & Not Listed \\
Carterocephalus palaemon & Chequered Skipper & Hesperiidae & S & S & 604& NA & Univoltine & AugustBefore & Not Listed \\
Coenonympha pamphilus & Small Heath & Nymphalidae & G & S & 3058 & NA &Univoltine & April & Vulnerable \\
Coenonympha tullia & Large Heath & Nymphalidae & S & S & 1409 & 44 &Univoltine & April & Endangered \\
Cupido minimus & Small Blue & Lycaenidae & G & S & 1453 & 89 &Univoltine & JulyBefore & Near Threatened \\
Erynnis tages & Dingy skipper & Hesperiidae & G & S & 856 & 85 &Univoltine & AugustBefore & Not Listed \\
Euphydryas aurinia & Marsh Fritillary & Nymphalidae & G & S & 1926 & NA & Univoltine & May & Vulnerable \\
Fabriciana adippe & High Brown Fritillary & Nymphalidae & G & S & 994 &NA & Univoltine & May & Endangered \\
Favonius quercus & Purple hairstreak & Lycaenidae & G & S & 494 & 83 &Univoltine & May & Not Listed \\
Gonepteryx rhamni & Brimstone & Pieridae & G & S & 483 & 81 & Univoltine& June & Not Listed \\
Hamearis lucina & Duke of Burgundy & Riodinidae & S & S & 951 & NA &Univoltine & JulyBefore & Vulnerable \\
Hesperia comma & Silver-spotted Skipper & Hesperiidae & S & S & 646 &103 & Univoltine & June & Vulnerable \\
Hipparchia semele & Grayling & Nymphalidae & S & G & 1654 & NA &Univoltine & April & Endangered \\
Limenitis camilla & White Admiral & Nymphalidae & S & S & 761 & 97 &Univoltine & May & Vulnerable \\
Maniola jurtina & Meadow brown & Nymphalidae & G & G & 4730 & 70 &Univoltine & April & Not Listed \\
Melanargia galathea & Marbled white & Nymphalidae & G & G & 2169 & 101 &Univoltine & May & Not Listed \\
Melitaea athalia & Heath Fritillary & Nymphalidae & G & G & 1619 & NA &Univoltine & March & Endangered \\
Melitaea cinxia & Glanville Fritillary & Nymphalidae & S & S & 579 & NA& Univoltine & March & Endangered \\
Nymphalis polychloros & Large Tortoiseshell & Nymphalidae & G & G & 197& NA & Univoltine & June & Regionally Extinct \\
Ochlodes sylvanus & Large skipper & Hesperiidae & G & G & 1029 & NA &Univoltine & May & Not Listed \\
Papilio machaon & Swallowtail & Papilionidae & S & S & 166 & 54 &Univoltine & JulyBefore & Vulnerable \\
Plebejus argus & Silver-studded Blue & Lycaenidae & S & G & 7105 & NA &Univoltine & May & Vulnerable \\
Polyommatus coridon & Chalk Hill Blue & Lycaenidae & G & S & 19584 & NA& Univoltine & June & Vulnerable \\
Pyrgus malvae & Grizzled Skipper & Hesperiidae & G & G & 1285 & 108 &Univoltine & July & Vulnerable \\
Pyronia tithonus & Gatekeeper & Nymphalidae & G & S & 2313 & 125 &Univoltine & June & Not Listed \\
Satyrium pruni & Black Hairstreak & Lycaenidae & S & S & 377 & NA &Univoltine & May & Endangered \\
Satyrium w-album & White-letter Hairstreak & Lycaenidae & S & S & 346 &NA & Univoltine & May & Vulnerable \\
Speyeria aglaja & Dark Green Fritillary & Nymphalidae & G & S & 1028 &51 & Univoltine & May & Near Threatened \\
Thecla betulae & Brown Hairstreak & Lycaenidae & G & S & 165 & NA &Univoltine & June & Vulnerable \\
Thymelicus acteon & Lulworth Skipper & Hesperiidae & S & S & 544 & NA &Univoltine & May & Near Threatened \\
Thymelicus lineola & Essex skipper & Hesperiidae & G & G & 707 & NA &Univoltine & May & Not Listed \\
Thymelicus sylvestris & Small skipper & Hesperiidae & G & G & 918 & 98 &Univoltine & May & Not Listed \\

\end{longtable}
\end{landscape}

\newpage

\renewcommand{\thetable}{S2} % makes this table Table S2

\begin{table}[h]
\centering
% Three equal-width, centered columns for –, 0, +
\begin{tabular}{l
                >{\centering\arraybackslash}p{2cm}
                >{\centering\arraybackslash}p{2cm}
                >{\centering\arraybackslash}p{2cm}}
\toprule
& \multicolumn{3}{c}{\textbf{Size vs Late-Larval Temperature Response}} \\
\cmidrule(lr){2-4}
\textbf{UK Red List Status} & \textbf{–} & \textbf{0} & \textbf{+} \\
\midrule
Unlisted      & 10 & 12 & 4  \\
Red Listed    & 5  & 9  & 12 \\
\bottomrule
\end{tabular}
\caption{Categorisation of Temperatures – Size trends by British Red list status (Unlisted versus either \emph{Regionally Extinct}, \emph{Endangered}, \emph{Vulnerable}, or \emph{Near-Threatened}) Temperature responses were categorised into positive, negative or not-detectable based on whether the 90\% posterior support of that species’ fitted coefficient straddles zero. ($\chi^2$-test with Yates' continuity correction $\chi^2$= 6.0952, df = 2, p-value = 0.047)}

\end{table}
%

\newpage

\section{SI Figures}

\renewcommand{\thefigure}{\textbf{S\arabic{figure}}}
\renewcommand{\figurename}{\textbf{Figure}}

\begin{figure}[H]
    \centering
    \includegraphics[width=\linewidth]{../figs/SI_corr_inTempLatYear.png}
    \caption{
        \textbf{Distribution of within species-level correlations between key predictors in the records in the NHM dataset.} Red line shows 0, blue line shows median. There is no consistent geographic trend through time. Temperature tended to be positively correlated with year, but the trend is very noisy. Likewise, there is an overall negative correlation between temeprature and latitude, but is neither strong or consistent. 
    }
\end{figure}



\begin{figure}[H]
    \centering
    \includegraphics[width=\linewidth]{../figs/SI_correlation_in_predictors.png}
    \caption{
        \textbf{Associations between cross-species predictors.}
        (A) Correlations among the three principal predictors.
        (B–D) Relationships between taxonomic family and other species-level predictors.
        Families vary substantially in average size, but are otherwise approximately evenly distributed across predictor space.
    }
\end{figure}

\begin{figure}[H]
    \centering
    \includegraphics[width=\linewidth]{../figs/Corr_sample_size_trends_NHM.png}
    \caption{
        \textbf{ Relationship between sample size, confidence of posterior estimate and identification of species-level trends in the NHM dataset.} Precision of posterior estimates was strongly related to the sample size for each species, but the identification of confident trends was less dependent on sample size. 
    }
\end{figure}

\begin{figure}[H]
    \centering
    \includegraphics[width=\linewidth]{../figs/PredictorsFigure_Both_withlabels.png}
    \caption{
        \textbf{Main text Figure 4 with species labels. } 
    }
\end{figure}


\begin{figure}[H]
    \centering
    \includegraphics[width=\linewidth]{../figs/compare_coefs_across_datasets.png}
    \caption{
        \textbf{Coefficient Correlations between datasets.} Here, each dot represents a species best-fit coefficent of the record-predictors within the two datasets. Error bars show 90\% posterior confidence interval. Estimates from the smaller dataset were mich more uncertain (note the unequal axis scales). Pearson's correlation of the mean estimates between the datasets was insigificant in each case ( p-values: 0.158, 0.429, 0.0797 respectively (unadjusted for multiple tests)). 
        }
\end{figure}



\newpage

\section{Specimen Spatio-temporal Distributions}


\begin{figure}[H]
\includegraphics[width=\linewidth]{../figs/SI_NHM_dataspread_1.png}
\end{figure}
\begin{figure}[H]
\includegraphics[width=\linewidth]{../figs/SI_NHM_dataspread_2.png}
\end{figure}
\begin{figure}[H]
\includegraphics[width=\linewidth]{../figs/SI_NHM_dataspread_3.png}
\end{figure}
\begin{figure}[H]
\includegraphics[width=\linewidth]{../figs/SI_NHM_dataspread_4.png}
\end{figure}
\begin{figure}[H]
\includegraphics[width=\linewidth]{../figs/SI_NHM_dataspread_5.png}
\end{figure}
\begin{figure}[H]
\includegraphics[width=\linewidth]{../figs/SI_NHM_dataspread_6.png}
\end{figure}
\begin{figure}[H]
\includegraphics[width=\linewidth]{../figs/SI_NHM_dataspread_7.png}
\end{figure}
\begin{figure}[H]
\includegraphics[width=\linewidth]{../figs/SI_NHM_dataspread_8.png}
\end{figure}

\textbf{Figure S5: Distribution of records (post-filtering) from the NHM dataset for each species}

% Ox Data Distributions
\newpage

\includegraphics[width=\linewidth]{../figs/SI_OUM_dataspread_1.png}
\begin{figure}[H]
\includegraphics[width=\linewidth]{../figs/SI_OUM_dataspread_2.png}
\end{figure}
\begin{figure}[H]
\includegraphics[width=\linewidth]{../figs/SI_OUM_dataspread_3.png}
\end{figure}
\begin{figure}[H]
\includegraphics[width=\linewidth]{../figs/SI_OUM_dataspread_4.png}
\end{figure}

\textbf{Figure S6: Distribution of records (post-filtering) from the NHM dataset for each species} NB Locations on the map have been slightly jittered to reduce overlap. 


\end{document}